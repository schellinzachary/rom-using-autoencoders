% !TEX root = master.tex

\usepackage[english]{babel}

\usepackage{dblfloatfix} 
\usepackage{dsfont} %% for \1
\usepackage{exscale}
\usepackage{mathrsfs} %mathscr{}
\usepackage[T1]{fontenc}
\usepackage[utf8]{inputenc}
\usepackage{lmodern}
\usepackage{graphicx}
\usepackage[utf8]{inputenc}
\usepackage{footnote}
\usepackage{hyperref}
\usepackage{fullpage}
\usepackage{amsmath,amssymb,amsfonts}
\usepackage{multicol}
\usepackage{wrapfig}
\usepackage{float}
\usepackage[ruled,vlined]{algorithm2e}
\usepackage{algorithmic}
\usepackage{subcaption}
\usepackage{pgfplots}
\usepackage{pgfplotstable}
\usepackage{tikz}
\usepackage{url}
\usepackage{dirtytalk}
\usepackage{gensymb}
\usepackage{siunitx,longtable}
\usetikzlibrary{external,positioning,backgrounds}
\tikzexternalize[prefix=Figures/]
\usetikzlibrary{shapes,arrows,snakes,chains}
\usetikzlibrary{matrix}
\usetikzlibrary{decorations}
\pgfplotsset{compat=newest}
\usepgfplotslibrary{groupplots}
\usepackage{tikzscale}
\usepackage{cleveref}
\usepackage{booktabs}
\usepackage{graphicx}
\usepackage[font={footnotesize}]{caption}
\usepackage{tocbibind}
\bibliographystyle{JHEP}

%%%%%%%%%%%%%%%%%%%%New Commands%%%%%%%%%%%%%%%%%%%%%%%%%%%%
%misc
\DeclareMathOperator*{\cumu}{\Epsilon} %cumulative energy POD
\DeclareMathOperator*{\base}{\Phi} %basis as matrix or vector
\DeclareMathOperator*{\snp}{u} %snapshot as matrix or vector
\DeclareMathOperator*{\mbs}{\kappa} %snapshot as matrix or vector

%BGK and rarefaction levels
\DeclareMathOperator*{\Kn}{\mathit{Kn}}
\DeclareMathOperator*{\hy}{\textbf{\textsf{H}}}
\DeclareMathOperator*{\rare}{\textbf{\textsf{R}}}

% intrinsic variables
\DeclareMathOperator*{\idhy}{\textbf{\textsf{h}}}
\DeclareMathOperator*{\idrare}{\textbf{\textsf{r}}}
\DeclareMathOperator*{\a1}{\alpha_{1}}
\DeclareMathOperator*{\a2}{\alpha_{2}}
\DeclareMathOperator*{\a3}{\alpha_{3}}

%Metric
\DeclareMathOperator*{\L2}{L_2}

% Neural networks
\DeclareMathOperator*{\frepar}{\theta} %set of weights and biases
\DeclareMathOperator*{\grd}{\mathbf{g}} %gradient for backprop
\DeclareMathOperator*{\btch}{P} %batch is matrix or tensor
\DeclareMathOperator*{\minib}{P_i} %minibatch is matrix or tensor
\DeclareMathOperator*{\btchfc}{P_{FCNN}} %batch spec. FCNN
\DeclareMathOperator*{\btchcn}{P_{CNN}} %batch spec. CNN
\DeclareMathOperator*{\nla}{\sigma} %non linear activation

%math convenience
\newcommand{\colvec}[1]{\ensuremath{\begin{pmatrix}#1\end{pmatrix}}}
\DeclareMathOperator*{\argmin}{\arg\!\min} %eckhard young theorem


%%%%%%%%%%%%%%%%%Tikz Styles%%%%%%%%%%%%%%%%%%%%%%%%%%%%%%%%%

\tikzstyle{block} = [rectangle, draw, 
text width=5em, text centered, rounded corners, minimum height=4em]
\tikzstyle{cube} = [rectangle, minimum width=1cm, minimum height=1cm, draw=black]
\tikzstyle{rec} = [rectangle,minimum height=4em,text centered, fill=blue!20,draw=black]
\tikzstyle{circ} = [circle,minimum size =0.4cm,text centered, draw=black,inner sep = 0pt]
\tikzstyle{arrow} = [thick,->,>=stealth]

%%%%%%%%%%%%%%%%%%%%%%%Colors%%%%%%%%%%%%%%%%%%%%%%%%%%%%%%%%
\definecolor{color0}{rgb}{0.12156862745098,0.466666666666667,0.705882352941177}
%%%%%%%%%%%header einstellungen%%%%%%%%%%%%%%%%%%%
\usepackage{fancyhdr}
%\pagestyle{fancy}
% \fancyhead{}
% \fancyhead[ER]{\rightark}
% \fancyhead[OL]{\leftmark} 

%\renewcommand{\sectionmark}[1]{\markboth{#1}{#1}} % set the \leftmark % set the \leftmark
%\fancyhf{}
%\fancyhead[EL,OR]{\thepage}
%\fancyhead[ER]{\rightmark}
%\fancyhead[OL]{\leftmark} 

%%%%%%%%%%%%%%%%%%%chapter logo %%%%%%%%%%%%%%%%%%%%%%%%%%5
\usepackage{titlesec,titletoc}



%%%%Keine Identierung
\setlength\parindent{0pt}

%some extra space for tables and lists with alpha numeric labeling
\renewcommand{\labelenumi}{(\alph{enumi})}
\setlength\extrarowheight{1.5pt}




