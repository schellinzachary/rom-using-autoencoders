\documentclass[Nike]{tuberlinbeamer}

\usepackage[ngerman,english]{babel}  % 'babel' muss geladen werden
\usepackage[utf8]{inputenc}  % optional, aber empfehlenswert
\usepackage{
	tikz,
	pgfplots,
	xparse,
	boxedminipage2e}
\usepackage[font={footnotesize}]{caption}
\usetikzlibrary{
	shapes,
	arrows,
	snakes,
	chains}
\tikzstyle{arrow} = [thick,->,>=stealth]

% Die ueblichen Angaben
\title{Model Order Reduction of Rarefied Gases Using Neural Networks}
\subtitle{Institut für Numerische Fluiddynamik}
\author[Zachary Schellin]{Zachary Schellin}
\institute{Technische Universität Berlin}

% Eigenes Logo einfuegen:
\renewcommand{\pathtomylogo}{csm_cfd_logo_invers_ohne_schrift_blauer_hintergrund_3fa262aacf.png}

\begin{document}

\begin{frame}
\maketitle
\end{frame}


\begin{frame}
\tableofcontents
\end{frame}
\section{Introduction}
\begin{frame}[fragile]{Introduction}
 hallo
\end{frame}
\section{The BGK-Model}
\begin{frame}[fragile]{The BGK-Model}
	\begin{boxedminipage}{.4\textwidth}
		\centering
		The BGK transport equation
		\begin{equation}
		\partial_t f + v \partial_x f = \frac{1}{\tau} (M_f - f)
		\end{equation}
	\end{boxedminipage}%
\hspace{1.5cm}
	\begin{boxedminipage}{.45\textwidth}
		\centering
		The Maxwellian distribution $M_f$
		\begin{equation}
		M_f = \frac{\rho(x,t)}{(2\pi R T(x,t))^{\frac{3}{2}}}\exp(-\frac{(v - u(x,t))^2}{2 R T(x,t)}) 
		\end{equation}
	\end{boxedminipage}
\vspace{2cm}
	\begin{boxedminipage}{.4\textwidth}
		\centering
		The relaxation time $\tau$
		\begin{equation}
		\tau^{-1} = \frac{\rho(x,t)T^{1-\nu}(x,t)}{Kn}
		\end{equation}
	\end{boxedminipage}%
\hspace{1cm}
	\begin{boxedminipage}{.33\textwidth}
		\centering
		The Knudsen number $Kn$
		\begin{equation}
		Kn = \frac{\lambda}{l}
		\end{equation}
	\end{boxedminipage}
	\begin{figure}[H]
		\begin{tikzpicture}
\begin{axis}[
    y=5cm,            % y unit vector
    hide y axis,        % hide the y axis
    xmode = log,        % logarithmic x axis
    axis x line*=bottom,% only show the bottom x axis line, without an arrow tip
    xmin=1e-3, xmax=1e2,% range for the x axis
    xlabel = Kn,
    width=\textwidth
]
\addplot [draw=none] coordinates {(0.001,0.9)};
\draw [decorate,decoration={brace,amplitude=4pt}] (axis cs:0.0008,.8) -- (axis cs: 0.0045,.8) node[rectangle split,rectangle split parts=2,midway,above=6pt] {Euler \nodepart{second} equations};
\draw [decorate,decoration={brace,amplitude=4pt}] (axis cs:0.0051,.8) -- (axis cs: 0.025,.8) node[rectangle split,rectangle split parts=2,midway,above=6pt] {NSF \nodepart{second} equations};
\draw [decorate,decoration={brace,amplitude=4pt}] (axis cs:0.035,.8) -- (axis cs: 1.4,.8) node[rectangle split,rectangle split parts=2,midway,above=6pt] {Transition  \nodepart{second} regime};
\draw [decorate,decoration={brace,amplitude=4pt}] (axis cs:1.5,.8) -- (axis cs: 14,.8) node[rectangle split,rectangle split parts=2,midway,above=6pt] {Kinetic  \nodepart{second} regime};
\draw [decorate,decoration={brace,amplitude=4pt}] (axis cs:15,.8) -- (axis cs: 150,.8) node[rectangle split,rectangle split parts=2,midway,above=6pt] {Free  \nodepart{second} flight};
%\draw [<->] (axis cs:0.001,-.8) -- (axis cs: 0.006,-.8) node[midway,fill=white] {Equilibrium};
\end{axis}
\end{tikzpicture}
		\caption{Partitioning of $Kn$, the Knudsen number, into levels of rarefaction.}
		\label{Fig:ExpKN}
	\end{figure}
\end{frame}

\section{Sod's shock tube}
\section{Proper Orthogonal Decomposotion (POD)}
\section{Neural Networks}
\section{Results}

\begin{frame}[fragile]{Verwendung der \texttt{tuberlinbeamer}-Klasse}
Es folgen demnächst ein paar Folien zur Verwendung dieser Dokumentklasse.
\begin{itemize}
\item Kenntnis der \emph{beamer}-Klasse ist von Vorteil
\end{itemize}
\end{frame}

\section{Discussion}

\begin{frame}{ToDo}
\begin{itemize}
\item \emph{ToDo} schreiben
\item \emph{ToDo} abarbeiten
\end{itemize}
\end{frame}

\end{document}
