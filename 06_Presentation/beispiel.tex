\documentclass[Nike]{tuberlinbeamer}

\usepackage[ngerman,english]{babel}  % 'babel' muss geladen werden
\usepackage[utf8]{inputenc}  % optional, aber empfehlenswert
\usepackage{
	tikz,
	pgfplots,
	xparse,
	boxedminipage2e,
	amsmath,
	hyperref,
	tocbibind
	}
\bibliographystyle{JHEP}
\usepackage[font={footnotesize}]{caption}
\usetikzlibrary{
	shapes,
	arrows,
	snakes,
	chains}
\tikzstyle{arrow} = [thick,->,>=stealth]

% Die ueblichen Angaben
\title{Model Order Reduction of Rarefied Gases Using Neural Networks}
\subtitle{Institut f\"ur Numerische Fluiddynamik}
\author[Zachary Schellin]{Zachary Schellin}
\institute{Technische Universität Berlin}

% Eigenes Logo einfuegen:
\renewcommand{\pathtomylogo}{csm_cfd_logo_invers_ohne_schrift_blauer_hintergrund_3fa262aacf.png}

\begin{document}

\begin{frame}
\maketitle
\end{frame}


\begin{frame}
\tableofcontents
\end{frame}
\section{Introduction}
\begin{frame}[fragile]{Introduction}
 hallo
\end{frame}
\section{The BGK-Model}
\begin{frame}[fragile]{The BGK-Model}
	\begin{itemize}
		\item The Boltzmann equation approximated by $\boldsymbol{Q}$ the BGK operator as a source term with
		\begin{equation}
		\partial_t f + v \partial_x f = \overbrace{\frac{1}{\tau} (M_f - f)}^{Q}
		\end{equation}

		\item The equilibrium solution is a Maxwellian distribution $\boldsymbol{M_f}$ with
		\begin{equation}
		M_f = \frac{\rho(x,t)}{(2\pi R T(x,t))^{\frac{3}{2}}}\exp(-\frac{(v - u(x,t))^2}{2 R T(x,t)}) 
		\end{equation}

		\item The duration to evolve into equilibrium is given by the relaxation time $\boldsymbol{\tau}$ with
		\begin{equation}
		\tau^{-1} = \frac{\rho(x,t)T^{1-\nu}(x,t)}{Kn}
		\end{equation}

		\item The rarefaction level is defined over the Knudsen number $\boldsymbol{Kn}$ with
		\begin{equation}
		Kn = \frac{\lambda}{l}
		\end{equation}
	\end{itemize}\footnote{\cite{BGK}}
\end{frame}
\begin{frame}[fragile]{The BGK-Model}
		\begin{figure}[H]
		\begin{tikzpicture}
\begin{axis}[
    y=3cm,            % y unit vector
    hide y axis,        % hide the y axis
    xmode = log,        % logarithmic x axis
    axis x line*=bottom,% only show the bottom x axis line, without an arrow tip
    xmin=1e-3, xmax=1e2,% range for the x axis
    xlabel = Kn,
    width=\textwidth,
]
\addplot [no markers, line width=6pt] table {%
0.002 1
0.003 1
};
\draw [|<->|,dashed] (axis cs:0.003,1) -- (axis cs: 0.01,1) node[midway,above] {Eueler};
\end{axis}
%\begin{scope}[decoration=brace]
%	\draw [decorate] (current axis.south-|0.003) -- (current axis.south-|0.01) node[midway,above] {Eueler};
%\end{scope}
%\draw [] (axis cs:0.25,2.1) node[above] {\(T\)};
%\begin{scope}[decoration=brace]
%\pgfdecorationsegmentamplitude=5pt
%\draw[decorate] (T2.south east) -- (T0.south west) node[midway,below=\pgfdecorationsegmentamplitude] {Part 1};
%\draw[decorate] (T14.south east) -- (T3.south west) node[midway,below=\pgfdecorationsegmentamplitude] {Part 2};
%\draw[decorate] (T20.south east) -- (T15.south west) node[midway,below=\pgfdecorationsegmentamplitude] {Part 3};
%\end{scope}
\end{tikzpicture}
		\caption{Partitioning of $Kn$, the Knudsen number, into levels of rarefaction.}
		\label{Fig:ExpKN}
	\end{figure}
\end{frame}


\section{Sod's shock tube}
\section{Proper Orthogonal Decomposotion (POD)}
\section{Neural Networks}
\section{Results}

\begin{frame}[fragile]{Verwendung der \texttt{tuberlinbeamer}-Klasse}
Es folgen demnächst ein paar Folien zur Verwendung dieser Dokumentklasse.
\begin{itemize}
\item Kenntnis der \emph{beamer}-Klasse ist von Vorteil
\end{itemize}
\end{frame}

\section{Discussion}

\begin{frame}{ToDo}
\begin{itemize}
\item \emph{ToDo} schreiben
\item \emph{ToDo} abarbeiten
\end{itemize}
\end{frame}
\bibliography{./other/references.bib}
\end{document}
